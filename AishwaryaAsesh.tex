% HW Template for CS 6150, taken from https://www.cs.cmu.edu/~ckingsf/class/02-714/hw-template.tex
%
% You don't need to use LaTeX or this template, but you must turn your homework in as
% a typeset PDF somehow.
%
% How to use:
%    1. Update your information in section "A" below
%    2. Write your answers in section "B" below. Precede answers for all 
%       parts of a question with the command "\question{n}{desc}" where n is
%       the question number and "desc" is a short, one-line description of 
%       the problem. There is no need to restate the problem.
%    3. If a question has multiple parts, precede the answer to part x with the
%       command "\part{x}".
%    4. If a problem asks you to design an algorithm, use the commands
%       \algorithm, \correctness, \runtime to precede your discussion of the 
%       description of the algorithm, its correctness, and its running time, respectively.
%    5. You can include graphics by using the command \includegraphics{FILENAME}
%
\documentclass[11pt]{article}
\usepackage{tikz}
\usetikzlibrary{positioning}
\newdimen\nodeDist
\nodeDist=20mm
\usepackage{amsmath,amssymb,amsthm}
\usepackage{graphicx}
\usepackage[margin=1in]{geometry}
\usepackage{fancyhdr}
\setlength{\parindent}{0pt}
\setlength{\parskip}{5pt plus 1pt}
\setlength{\headheight}{13.6pt}
\newcommand\question[2]{\vspace{.25in}\hrule\textbf{#1: #2}\vspace{.5em}\hrule\vspace{.10in}}
\renewcommand\part[1]{\vspace{.10in}\textbf{(#1)}}
\newcommand\algorithm{\vspace{.10in}\textbf{Algorithm: }}
\newcommand\correctness{\vspace{.10in}\textbf{Correctness: }}
\newcommand\runtime{\vspace{.10in}\textbf{Running time: }}
\pagestyle{fancyplain}
\lhead{\textbf{\NAME\ (\UID)}}
\chead{\textbf{HW\HWNUM}}
\rhead{CS 6350, \today}
\begin{document}\raggedright
%Section A==============Change the values below to match your information==================
\newcommand\NAME{Aishwarya Asesh}  % your name
\newcommand\UID{u1063384}     % your utah UID
\newcommand\HWNUM{2}              % the homework number
%Section B==============Put your answers to the questions below here=======================

% no need to restate the problem --- the graders know which problem is which,
% but replacing "The First Problem" with a short phrase will help you remember
% which problem this is when you read over your homeworks to study.


\question{1}{Feature Expansion}
According to question:
\begin{equation}
f_r(x_1, x_2) = \left\{
    \begin{array}{rl}
      +1 & 4x_1^4 + 16x_2^4 \leq r;\\
      -1 & \mbox{otherwise}
    \end{array}
\right.
\label{eq:f_r}
\end{equation}

Construct a function $\phi(x_1, x_2)$ to change ($x_1,x_2$) = ($x_{1}^4,x_{2}^4$) will change the whole equation to linearly separable equation. 
\\ Thus we can conclude by writing the following equation:
\\
\begin{equation}
\phi_r(x_{1}^4,x_{2}^4) = \left\{
    \begin{array}{rl}
      +1 & 4x_1^4 + 16x_2^4 \leq r;\\
      -1 & \mbox{otherwise}
    \end{array}
\right.
\label{eq:f_r}
\end{equation}
 
where $w^{T}$$\phi_r(x_{1}^4,x_{2}^4)$$>=c$
\\$w^{T}$=[4,16] and c=r



\question{2}{Mistake Bound Model of Learning} 
\part{1}Here as we can observe, radius is mentioned as an integer.
\\total number of functions that exist is dependent on values of r.
\\ Since we know size of concept class C = total no. of func.
\\ So here size of concept class is 80\\
\part{2} Algorithm will only make a mistake when label is not predicted as defined by the function.
\\Derived Label * Actual Label $<=$0 then algoithm will make a mistake
\\or if ($x_1^2+x_2^2-r^2$)$\times$ $y$$\leq 0$
\\then there is a mistake\\
\part{3} Pseudocode for given statement
\\If there is a mistake,
\\update r with  integer($\sqrt{x^2+y^2+1}$)
\\Thus the $(x_1,x_2)$ pair gets the required label.\\
\part{4}Mistake driven learning algo to learn the function
\\1. $(x_1,x_2)$ pair (taken from domain of previous answers).
\\2. Lets consider radius = r = 1
\\3. For the pair input taken in step 1, if the label output ($x_1^2+x_2^2-r^2$) is positive, then we can say that the pair $(x_1,x_2)$ lies inside the range (circle).
\\4. Get next label
\\5. Get the value of (new or derived label) *(original or old label)
\\6. If the value in previous step is positive, then there is no mistake.
\\7. Repeat the above steps for further values.
\\As we see the maximum no. of mistakes that algo can make is 79.
\\This situation will occur only when sorted input of co-ordinates is provided to the algorithm.
\\Value of r will get updated on every single step in the above assumed case.\\
\part{5a} By storing the labels and radius, we can overcome this problem\\
\part{5b} If most number of the functions present in concept class will predict a label (new label) that is different from the original label, we can state that a mistake has occured.\\
\part{5c}Pseudocode
1. Let the radius be r=1.
\\2. $(x_1,x_2)$ is the input pair.
\\3. The total number of functins contained in concept class is ($x_1^2+x_2^2=r^2$)
\\4. For a pair of valid input we need to verify if positive or negative label is predicted.Most number of functions with similar output is taken into consideration and derived label (new label) is considered as the majority value.
\\5. If predicted label is not matching with the original label or old label (y), the func. that predicted a wrong label are removed.
\\6. Repeat the above steps and stop when only a single function is left in concept class.
\\[10pt] Mistake bound value = log(80) 
\question{3}{Perceptron Algorithm and its Variants - EXPERIMENTS}
\part{1} New Weight Vector
\\$[\ \ 0, \ \ 1, \ \ 0, \ \ -1, \ \ 2 \ \ ]$
\\4 is the number of mistakes\\
\part{2} Weight vector is random ranging from values -2 and 2.\\ Bias is also random ranging from -2 and 2. 
\\Learning rates taken = 1 and 0.1
\\FOR LEARNING RATE 1
\\Mistakes SIMPLE PERCEPTRON = 1445
\\Mistakes MARGIN PERCEPTRON (1) = 1520
\\SIMPLE PERCEPTRON
\\TEST ON TRAIN SET 
\bgroup 
\def\arraystretch{1.2}
\begin{tabular}{|c|} \hline 
{\bf \underline {ACCURACY}} \\ \hline
76.78 \\ \hline
\end{tabular}
\egroup
\\TEST ON TEST SET
\bgroup 
\def\arraystretch{1.2}
\begin{tabular}{|c|} \hline 
{\bf \underline {ACCURACY}} \\ \hline
77.76  \\ \hline
\end{tabular}
\egroup
\\[15pt]MARGIN PERCEPTRON
\\TEST ON TRAIN SET 
\bgroup 
\def\arraystretch{1.2}
\begin{tabular}{|c|} \hline 
{\bf \underline {ACCURACY}} \\ \hline
81.96  \\ \hline
\end{tabular}
\egroup
\\TEST ON TEST SET
\bgroup 
\def\arraystretch{1.2}
\begin{tabular}{|c|} \hline 
{\bf \underline {ACCURACY}} \\ \hline
81.74 \\ \hline
\end{tabular}
\egroup

\newpage
FOR LEARNING RATE 0.1
\\Mistakes SIMPLE PERCEPTRON = 1414
\\Mistakes MARGIN PERCEPTRON (1) = 1957
\\SIMPLE PERCEPTRON
\\TEST ON TRAIN SET 
\bgroup 
\def\arraystretch{1.2}
\begin{tabular}{|c|} \hline 
{\bf \underline {ACCURACY}} \\ \hline
71.42 \\ \hline
\end{tabular}
\egroup
\\TEST ON TEST SET
\bgroup 
\def\arraystretch{1.2}
\begin{tabular}{|c|} \hline 
{\bf \underline {ACCURACY}} \\ \hline
70.82  \\ \hline
\end{tabular}
\egroup
\\[15pt]MARGIN PERCEPTRON
\\TEST ON TRAIN SET 
\bgroup 
\def\arraystretch{1.2}
\begin{tabular}{|c|} \hline 
{\bf \underline {ACCURACY}} \\ \hline
77.90  \\ \hline
\end{tabular}
\egroup
\\TEST ON TEST SET
\bgroup 
\def\arraystretch{1.2}
\begin{tabular}{|c|} \hline 
{\bf \underline {ACCURACY}} \\ \hline
77.01 \\ \hline
\end{tabular}
\egroup

\newpage

\part{3} Weight vector is random ranging from values -2 and 2.\\ Bias is also random ranging from -2 and 2. 
\\Below results are reported after shuffling the data
\\SINCE training is done on same "a5a.train" file, number of mistakes while training are same
\\TEST ON TRAIN SET
\\
\bgroup 
\def\arraystretch{1.5}
\begin{tabular}{|l|c|c||c|c|} \hline 
{\bf \underline {MARGIN}} & {\bf \underline {MISTAKES WHILE TRAINING}} & {\bf \underline {LEARNING RATE}} & {\bf \underline {EPOCH}} & {\bf \underline {ACCURACY}} \\ \hline
0 & 7223 & 1 & 3 & 74.38 \\ \hline
0 & 12147 & 1 & 5 & 72.70 \\ \hline
0 & 7079 & 0.1 & 3 & 75.41 \\ \hline
0 & 11741 & 0.1 & 5 & 71.53 \\ \hline
0 & 7023 & 0.01 & 3 & 72.13 \\ \hline
0 & 11602 & 0.01 & 5 & 72.84 \\ \hline

\end{tabular}
\egroup
\\[10pt]TEST ON TEST SET
\\
\bgroup 
\def\arraystretch{1.5}
\begin{tabular}{|l|c|c||c|c|} \hline 
{\bf \underline {MARGIN}} & {\bf \underline {MISTAKES WHILE TRAINING}} & {\bf \underline {LEARNING RATE}} & {\bf \underline {EPOCH}} & {\bf \underline {ACCURACY}} \\ \hline
1 & 7223 & 1 & 3 & 73.88 \\ \hline
1 & 12147 & 1 & 5 & 71.57 \\ \hline
1 & 7079 & 0.1 & 3 & 75.77 \\ \hline
1 & 11741 & 0.1 & 5 & 70.05 \\ \hline
1 & 7023 & 0.01 & 3 & 70.45 \\ \hline
1 & 11602 & 0.01 & 5 & 71.94 \\ \hline

\end{tabular}
\egroup
\\As we see due to random initialization of weight vector and bias and due to shuffling of data , the results vary by a large margin everytime we run the program. The results are however similar for epoch 3 and 5 in every run.
\newpage
\part{4} Weight vector is random ranging from values -2 and 2.\\ Bias is also random ranging from -2 and 2. 
\\NON - SHUFFLED DATA
\\TEST ON TRAIN SET
\\
\bgroup 
\def\arraystretch{1.5}
\begin{tabular}{|l|c|c||c|c|} \hline 
{\bf \underline {MARGIN}} & {\bf \underline {MISTAKES WHILE TRAINING}} & {\bf \underline {LEARNING RATE}} & {\bf \underline {EPOCH}} & {\bf \underline {ACCURACY}} \\ \hline
0 & 4466 & 1 & 3 & 76.50 \\ \hline
0 & 7463 & 1 & 5 & 77.23 \\ \hline

\end{tabular}
\egroup
\\[10pt]TEST ON TEST SET
\\
\bgroup 
\def\arraystretch{1.5}
\begin{tabular}{|l|c|c||c|c|} \hline 
{\bf \underline {MARGIN}} & {\bf \underline {MISTAKES WHILE TRAINING}} & {\bf \underline {LEARNING RATE}} & {\bf \underline {EPOCH}} & {\bf \underline {ACCURACY}} \\ \hline
0 & 4459 & 1 & 3 & 74.73 \\ \hline
0 & 7406 & 1 & 5 & 75.61 \\ \hline

\end{tabular}
\egroup
\newpage Weight vector is random ranging from values -2 and 2.\\ Bias is also random ranging from -2 and 2. 
\\SHUFFLED DATA
\\TEST ON TRAIN SET
\\
\bgroup 
\def\arraystretch{1.5}
\begin{tabular}{|l|c|c||c|c|} \hline 
{\bf \underline {MARGIN}} & {\bf \underline {MISTAKES WHILE TRAINING}} & {\bf \underline {LEARNING RATE}} & {\bf \underline {EPOCH}} & {\bf \underline {ACCURACY}} \\ \hline
1 & 11636 & 1 & 3 & 69.19 \\ \hline
1 & 12076 & 1 & 5 & 68.78 \\ \hline

\end{tabular}
\egroup
\\[10pt]TEST ON TEST SET
\\
\bgroup 
\def\arraystretch{1.5}
\begin{tabular}{|l|c|c||c|c|} \hline 
{\bf \underline {MARGIN}} & {\bf \underline {MISTAKES WHILE TRAINING}} & {\bf \underline {LEARNING RATE}} & {\bf \underline {EPOCH}} & {\bf \underline {ACCURACY}} \\ \hline
1 & 12076 & 1 & 3 & 68.83 \\ \hline
1 & 20187 & 1 & 5 & 68.70 \\ \hline

\end{tabular}
\egroup
\end{document}
